\usepackage{changepage, amsmath, amssymb, pgfplots, tikz, xcolor, chemfig, mhchem,tabularx,graphicx} % Hier werden alle packages geladen, die ich in diesem Dokument benutze. Ich definiere Farben und Befehle, die ich häufig verwende.
\usepackage{darkmode} % Dieses package wird nicht benutzt. Aber mit \enabledarkmode kann man die Papierfarbe auf schwarz, und die Schriftfarben und weiß invertieren.
\usepackage{fullpage} % Mit diesem Package wird die ganze Seite genutzt. 
\pgfplotsset{compat=1.18} 
\usetikzlibrary{3d, calc,angles,decorations.markings,quotes} 
\definecolor{jade}{HTML}{00A36C} % Grün
\definecolor{crimson}{HTML}{DC143C} % Rot
\definecolor{pinke}{HTML}{FF1493} % Pink
\definecolor{blaue}{HTML}{0000CD} % Blau
\definecolor{graue}{HTML}{c3cbc4} % Grau, selbsterklärend, schätze ich
\newcommand{\dreid}[7]{\begin{tikzpicture}
    \begin{axis}[
        xlabel={$#1$}, ylabel={$#2$},
        minor tick num=0,
        xmin=#3, xmax=#4, ymin=#5, ymax=#6,
        samples=100,
        axis y line=center, axis x line=middle,
        ]
        #7;
    \end{axis}
\end{tikzpicture}} % Für Graphen.
\usepackage[most]{tcolorbox}
\newtcolorbox{qq}[2][]{%
    lower separated=false,
    enhanced,
	boxrule=0.75pt,
	sharp corners,  % Square edges
	colframe=black,  % Set the color of the outline
	colback=white,  % Set the color of the fill
	coltext=black,
    top=0pt,
	#1,
    breakable,
} % Definiert die Boxen, die ich verwende
\usepackage{float} % Damit ich bilder einfügen kann
\newcommand{\bei}[2]{\textcolor{crimson}{\text{#1}}\text{ bei }\textcolor{crimson}{#2}} 
\newcommand{\boxx}[2]{\subsubsection{\jade{#1}}\begin{qq}{1cm}#2\end{qq}} % Definiert die Boxen, die ich verwende
\newcommand{\crimson}[1]{\textcolor{crimson}{#1}}
\newcommand{\jade}[1]{\textcolor{jade}{#1}}
\newcommand{\blaue}[1]{\textcolor{blaue}{#1}}
\newcommand{\boxxx}[1]{\begin{qq}{1cm}#1\end{qq}}
\newcommand{\boxxline}{\noindent\rule{\textwidth}{0.4pt}}
\usepgfplotslibrary{fillbetween}
\usepackage[hidelinks]{hyperref}
\renewcommand{\contentsname}{Inhaltsverzeichnis}
\newcommand{\minipagee}[2]{\begin{minipage}{#1\textwidth}#2\end{minipage}}
\newcommand*{\rom}[1]{\expandafter\@slowromancap\romannumeral #1@}